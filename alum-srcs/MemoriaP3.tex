\documentclass[11pt]{article}
%Gummi|065|=)
\usepackage[utf8]{inputenc}
\usepackage{graphicx}
\title{\textbf{Informática Gráfica}}
\author{Prácticas 1, 2 y 3\\
		Laura Gómez Garrido}
\date{}
\begin{document}

\maketitle

\section{Modelo jerárquico y lista de parámetros:}

Se ha realizado un muñeco que será capaz de realizar movimientos simples como mover sus brazos y piernas, además de girarse y alzarse. Todos los parámetros comparten los siguientes valores: $v_0=0.0001$, $\Delta=0.0001$ y $a=0.00000001$.
\begin{itemize}
\item \textbf{Hombros(x2):} Matriz de Rotación respecto al eje x acotada. Tiene de valor central 90, semiamplitud 70 y una frecuencia de 0.06. Es una MediaExtremidad.
\item \textbf{Codos(x2):} Matriz de Rotación respecto al eje x acotada. Tiene de valor central 50, semiamplitud 50 y una frecuencia de 0.06. Es una MediaExtremidad.
\item \textbf{Giro Cuerpo(x1):} Matriz de Rotación respecto al eje y acotada. Tiene de valor central 0, semiamplitud 70 y una frecuencia de 0.03. Se mueve todo el Bailarín.
\item \textbf{Levantar Cuerpo(x1):} Matriz de Traslación respecto al eje y acotada. Tiene de valor central 0, semiamplitud 2 y una frecuencia de 0.12. Se mueve todo el Bailarín.
\item \textbf{Pierna(x2):} Matriz de Rotación respecto al eje x acotada. Tiene de valor central -40, semiamplitud 40 y una frecuencia de 0.12. Es una MediaExtremidad.
\item \textbf{Rodilla(x2):} Matriz de Rotación respecto al eje x acotada. Tiene de valor central -50, semiamplitud 50 y una frecuencia de 0.12. Es una MediaExtremidad.
\end{itemize}

\newpage

\section{Grafo de escena}
\begin{center}
	\begin{figure}[h]
		\includegraphics[scale=0.6]{GrafoEscena}
	\end{figure}
\end{center}
\end{document}
